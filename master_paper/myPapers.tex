\begin{szuAppendix}{攻读硕士学位期间研究成果}
\begin{spacing}{1.5}
\begin{flushleft}
\songti\zihao{-3}{\textbf{论文}}
\end{flushleft}
\songti\zihao{5}
\vspace{0.3\baselineskip}

\noindent
[1] Zhou Ming-Yang, \textbf{Li Xiao-Yu}, Xiong Wen-Man, and Liao Hao, The coupling strength versus convergence speed in pinning control, Nonlinear Dynamics, 2019, pp. 1-12. (SCI论文,中科院小类1区,IF=4.3)

\smallskip
\noindent
[2] Zhou Ming-Yang, \textbf{Li Xiao-Yu}, Xiong Wen-Man, and Liao Hao, Coevolution of synchronization and cooperation in real networks, 2019,IJMPC (SCI论文)
%[1] Sen Jia, \textbf{Kuilin Wu}, Jiasong Zhu and Xiuping Jia. Spectral–Spatial Gabor Surface Feature Fusion Approach for Hyperspectral Imagery Classification. IEEE Transactions on Geoscience and Remote Sensing, 2019, 57(2): 1142–1154. (2018年影响因子:4.662)
%
%\smallskip
%\noindent
%[2] \textbf{Kuilin Wu} and Sen Jia. 2D Gabor-based Sparse Representation Classification for Hyperspectral Imagery. IEEE Digital Image Computing: Techniques and Applications (DICTA). 2018, 1-8. (EI收录)
%
%
%\smallskip
%\noindent
%[3] Sen Jia, \textbf{KuiLin Wu}, Meng Zhang and Jie Hu. Three-dimensional Surface Feature for Hyperspectral Imagery Classification. International Conference on Neural Information Processing (ICONIP). 2017, 270-278. (EI收录)



%\vspace{2\baselineskip}	
%\begin{flushleft}
%\songti\zihao{-3}{\textbf{专利}}
%\end{flushleft}
%\songti\zihao{5}
%\vspace{0.3\baselineskip}
%
%\noindent
%[1] 贾森,\textbf{吴奎霖},朱家松,邓琳. 高光谱遥感图像的特征提取方法及装置. 专利号:201710800249.4. (已实审)

\end{spacing}
\end{szuAppendix}
