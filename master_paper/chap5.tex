\section{总结与展望}
\subsection{本文总结}

本文主要研究网络重要节点检法及在牵制控制中的应用。
文章从大规模网络的控制问题出发,提出了耦合强度范围和收敛速度两个指标;通过对耦合强度范围的扰动分析,提出了一个重要节点检测算法,初步解决了牵制控制节点集合选择这一难题。本文的研究主要内容和创新点如下:

1.	耦合强度范围和收敛速度。
这一部分主要针对传统比率判据无法精确衡量网络牵制控制这一问题,提出两个评价指标:耦合强度范围和收敛速度。
耦合强度范围描述了网络模型中耦合强度满足的条件;收敛速度从控制稳定性的角度,描述了网络得到控制的速度。
一般而言,耦合强度范围越大,网络更容易得到控制;收敛速度越小,该系统中存在一个更小的李雅普诺夫指数,网络达到稳定的速度也就越快。
此外,研究发现不存在一个控制节点集合使传统判据、耦合强度范围和收敛速度这三个指标同时达到最优。
通过真实网络上的实验,验证了耦合强度范围和收敛速度的精确性和有效性。


2. 基于耦合强度范围扰动的重要节点检测算法。
这一部分主要针对于传统重要节点检测算法选择控制节点时效果不稳定这一问题,在耦合强度范围的基础上,通过扰动理论,提出了一个重要节点检测的贪婪算法。
该算法实现简单而且时间复杂度低,能快速地计算出网络中的重要节点,极大地增强节点对网络的控制能力。
通过6个真实网络上的性能分析,验证了基于耦合强度范围扰动算法的有效性和准确性。
此外,实验还验证了传统判据和耦合强度范围之间的强相关关系。

耦合强度范围和收敛速度这两个指标,理论上细化了网络牵制控制的概念,加深了对牵制控制的理解;基于耦合强度范围扰动的重要节点检测算法,在部署网络控制节点的应用方面,具有比较大的指导意义。


\subsection{研究展望}

为了进一步探究复杂网络的牵制控制问题,这里总结以下不足之处以便于后续的拓展研究:

1.	牵制控制能力可以划分为耦合强度范围和网络收敛速度。本文在耦合强度范围上着墨较多,但在网络收敛速度上没有进一步分析。收敛速度决定了网络到达稳定状态的快慢,这个指标可以将牵制控制和间歇控制结合起来。间歇控制是指在某个固定时间内使网络到达同步状态,其更易实现且耗能少。两者结合,网络收敛速度越大,间歇控制时间越短,网络响应越快,在工程应用上有很大的研究价值。

2.	本文基于耦合强度的扰动,提出了一个贪婪算法来选择网络的重要节点。然而,最佳牵制控制节点的选择是一个NP问题,可能会存在这样一种现象:5\%比例的最佳控制节点集合和10\%比例的最佳控制节点集合之间不是包含关系。有一些节点在控制节点数量较少时对网络控制能力较强,在控制节点增加时,部分节点可能不再是最佳控制节点。因此,该算法为了精确描述牵制控制节点集合,在细节上需要优化更多;比如说和蚁群算法相结合,选择蚁群信息素更浓郁的节点作为控制节点;此外还可以和模拟退火算法相结合,概率性地弹出和选择节点,从而得到更优的控制节点。

综上所述,希望通过对以上问题进一步的探索,完善网络科学在同步控制领域的研究,在工程上为网络重要节点选择以及牵制控制起到引导作用。

\clearpage
