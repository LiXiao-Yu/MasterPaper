\section{绪论}
\subsection{研究背景}
自然界中一大类网络,例如社交网络、生物学上的捕食网络 \cite{Ringma2018}等,可以抽象为由点和线组成的拓扑关系网络。
这些网络上的信息传播、疾病扩散、消息流等大多数行为一般是非线性的,无法采用简单的叠加原理来分析和解释,这造成了研究上的复杂性。因此,大多数研究都采用动力学分析、分形理论、控制理论等方法去解析这个非线性系统 \cite{Wang2017, 汪小帆2006}。
正是因为复杂网络具有普遍的适用性和兼容性,在统计物理学、计算机科学、传播学等诸多领域有极高的应用价值和现实意义。

在早期,复杂网络的研究主要在于规则网络和随机网络的性质构造方面。
随着计算机数据处理能力的提高,人们发现大部分真实网络不是规则网络和随机网络,它们具有不同的性质特征。
在BA无标度网络和WS小世界网络网络发现后,更多的研究开始关注于这些网络中的一些显著性现象,例如传播和同步 \cite{汪小帆2006, Liao2017, Holme2019}。
传播是指节点间发生信息交流,节点的某种状态被其他节点所影响而发生改变的现象,类比于现实生活中疾病的传播,某种舆论在社交媒体上的传播,或者计算机病毒的传播。同步更倾向于群体的某种一致性行为。1673年,从惠更斯发现钟摆的同步行为以来,生物学家也发现了自然界中,生物种群也常发生同步现象。例如,在夏季某时刻,聚在一起的萤火虫发光的频率是近似相同的 \cite{Brandner2016}。这种神秘的同步现象也引发了19世纪初期混沌动力学的热潮,其后描述同步行为的Kuramoto模型被提出,混沌同步的研究也愈发深入 \cite{Rodrigues2016}。

% 提出控制问题
然而,大多数网络无法到达同步状态。通常而言,这些网络需要人为地找到一些重要节点,对这些节点施加控制策略,从而使网络达到同步状态,这就是牵制控制\cite{Zhou2016}。牵制控制在自然界中也极其常见,例如蜂群搬家 \cite{Kennedy2006}。一般而言,在一个蜂巢的蜜蜂搬家过程中,并不是所有的蜜蜂都知道新家的具体位置的,大概只有3\%-5\%的“侦查蜂”知道新家的具体位置和路线,而搬家过程中,这小部分蜂群却能保证98\%的蜜蜂顺利到达新家而不走丢。在工业方面,牵制控制也用处广泛,例如在多智能体的系统(如传感器网络、机器人通信网络)中,研究人员只需要控制一小部分特定节点的状态,就能控制整个网络 \cite{He2017, Chen2017}。

网络牵制控制面临着一个难题——如何检测网络中的重要节点。
一般来说,网络重要节点的检测主要通过网络的拓扑结构,以节点中心性排序算法来选择重要节点,出现了度中心性、介数中心性、接近中心性和特征向量中心性等算法,在不同的工程领域有广泛的应用。
例如在搜索引擎中,网络重要节点检测问题等价于网页重要性排序问题。
通常而言,搜索引擎通常会通过用户输入的关键词,快速找到一系列的网页,通过对网页的重要性排序,准确给出用户所需要的内容。
但随着万维网呈指数倍的增长,更好的搜索引擎中网页重要性排序算法迫在眉睫。
1997年,李彦宏提出“超链分析技术”,该技术将网页类比于科技论文,通过科技论文被引用次数的多寡,来确定这篇论文的好坏。1998年,Lawrence等人共同开发了搜索引擎Google,并发明了PageRank专利。其将众多的网页视为节点,网页之间的超链接视为一条有向边,在这个复杂网络中,PageRank实现了网页的评级 \cite{Page1999}。

总体而言,节点重要性检测算法在复杂网络领域有着广泛的理论价值。
但在大规模网络中,很多算法在选择控制节点时,效果不稳定,不同网络中差异巨大。
事实上,选择恰当的牵制控制节点是一个NP-hard问题 \cite{Morone2015} ,在大多数情况下,研究人员只能找到一个较优的控制节点集合来实现牵制控制。
但恰当的牵制控制节点的选择,能够高效地确保系统的同步。因此,牵制控制近些年来也吸引了计算机科学、混沌物理学、统计学、生物学等广泛的关注。

\subsection{国内外研究现状}

\subsubsection{网络控制问题}
%虽然网络重要节点检测算法很多,不同方法有不同的特点和适用范围,但更重要的是,网络节点的重要性如何衡量。就目前而言,不存在一个公认的统一的标准来衡量网络节点的重要性。因此,本文从控制问题出发,以节点对网络的牵制控制能力大小作为指标,来研究节点的重要性,并比较不同方法的优缺点。

网络控制问题起源于网络同步问题。动力学系统除了自身的演化外,他们之间还存在相互作用(耦合),在满足某种条件时,这些系统的状态输出逐渐趋同进而完全相等,这种行为就是同步。复杂网络的同步可以大致分为两大类:完全同步和广义同步。完全同步更多的指的是相同单元的所有状态趋近于同步,广义同步更多的是指不同单元间的某种状态趋近于同步,例如频率,相位等状态 \cite{汪小帆2006}。

在惠更斯提出著名的钟摆同步后的几个世纪,复杂系统的混沌同步学吸引了学者们深入的研究。1967年,Winfree做了开创性的工作,他假设每个振子只与其周围的有限振子存在耦合(强相互力)作用,这样就忽略了振子的振幅,将网络中振子的同步问题简化成相位变化问题 \cite{Ariaratnam2001}。
Kuramoto 在前者的基础上提出,无论振子之间存在的耦合作用多么微弱,振子的动力学特性都可以用简单的动力学方程来模拟。并且,在全耦合网络中,只要耦合强度足够高,该网络的振子一定能到达同步的状态 \cite{Rodrigues2016}。
20世纪同步上的工作都在规则网络上展开,在BA无标度网络和WS小世界网络发现后,人们更关注于网络的拓扑结构对网络同步的影响。

同步可以优化复杂网络系统的功能,在通信系统、激光超导等方面应用上具有重大意义。
但在大部分情况下,复杂网络的同步很难到达稳定的状态,研究人员需要对网络节点施加人为的控制策略。
一般而言,控制的方法主要分为四类:包括牵制控制、脉冲控制、自适应控制、间歇控制等等。

牵制控制 \cite{Grigoriev1997}:通过控制复杂系统中的一小部分节点,从而有效抑制网络的时空混沌现象,使整个网络到达同步稳定状态的过程叫做牵制控制。牵制控制在生物学上很常见,例如秀丽隐杆线虫(Caenorhabditis elegans)的神经系统的控制。研究表明线虫的神经系统有约300个神经元和约2400条神经连接突起,生物学家发现通过刺激少数(约49个)神经元,就能控制线虫的整个神经系统,这些受控制的神经元仅占神经系统的17\% \cite{Chen2014}。 通常而言,网络规模越大,网络系统的结构也愈发复杂,控制网络中所有节点往往不合理且开销巨大。但可以采用牵制控制方法,控制少部分网络节点来控制整个网络。

牵制控制节点的选择成为了一个难题。2005年,Wang等人提出了无标度网络中两种常见的牵制控制节点选择策略 \cite{Wang2002}。一种是随机牵制控制策略,另一种则是依次选择网络中度最大的若干节点的特定牵制控制策略。顾名思义,前者在网络中随机地选择牵制控制节点,后者则通过节点的度从大到小排序,选择度较大的部分节点并对其施加牵制控制作用。同时,文献指出特定牵制控制策略比随机牵制控制策略,在无标度网络中更为有效一些。
在这之后,Song等人研究多智能体的一致性问题,基于拉萨尔不变集原理,讨论了牵制控制的种类和数量,并给出网络到达一致性状态的充分条件 \cite{Song2010}。
Xiong等人研究带马尔可夫跳的有向网络牵制控制问题,通过对网络结构的分析,提出了一个确保所有节点都达到同步的牵制控制方案,该方案还可以识别马尔可夫跳模型的最小牵制控制节点数 \cite{Xiong2010}。
Lu等人研究有向图的牵制控制问题,指出当且仅当牵制控制节点集合可以访问其他所有节点时,该有向网络可以被稳定到一些不稳定点轨迹 \cite{Lu2010}。

脉冲控制 \cite{Lin2016}:复杂系统中常出现一些突发性噪声,这些噪声可能会导致网络节点状态发生变化,这种情况下需要用到脉冲控制。脉冲控制是指在离散时间内对系统施加控制脉冲而使系统状态发生改变的非连续控制手段。脉冲控制可以使系统由混沌状态变成稳态,也可以由稳态变成混沌状态。
Wong等人研究具有延迟耦合和混合脉冲的复杂网络随机同步问题,通过使用平均脉冲间隔法和比较原理,推导了实现复杂网络指数同步的几类条件 \cite{Wong2013}。
Liu等人研究复杂网络在时间尺度的同步问题,将脉冲控制与牵制控制相结合,实现了CDNs在时间尺度上与孤立节点状态的同步 \cite{Liu2016}。

自适应控制 \cite{Ahmed2017}:自适应控制是指通过系统参数的自我调节来适应系统本身或系统所处环境噪声的一种控制手段。
Shi等人利用线性和自适应反馈控制研究复杂网络的集群同步问题,设计了两种不同且不要求非延迟和延迟耦合矩阵对称或不可约的控制器 \cite{Shi2017}。
Qin等人研究了具有多个时延的复杂网络鲁棒同步问题,考虑到大部分网络存在的环境噪声,开发了自适应控制器来保证此类网络的鲁棒同步 \cite{Qin2018}。
Zhang等人分析了复杂网络的指数同步问题,基于1-范数的分析,结合自适应控制和脉冲控制,得出了指数同步的充分条件 \cite{Zhang2018}。

间歇控制 \cite{Liu2016a}:间歇控制指在某个周期内非连续地对系统施加控制。与脉冲控制在时间点对系统施加脉冲不同,间歇控制通常在某个时间段内施加控制使系统达到稳定同步状态。通常而言,脉冲控制在实际应用中更易实现,而且耗能更少。
Liu等人通过固定周期的间歇控制研究线性耦合网络的集群同步问题,设计了一种间歇控制增益的集中式自适应算法 \cite{Liu2015}。
Wang等人通过间歇控制,对马尔可夫切换的耦合时滞系统进行了稳定性分析,结合李雅普诺夫方法和Kirchhoff矩阵树定理,得出了该系统间歇控制的稳定标准 \cite{Wang2018}。

\subsubsection{重要节点检测算法}

在牵制控制过程中,研究人员需要通过网络重要节点检测算法计算出重要节点,并对重要节点施加控制策略,从而使整个网络到达同步稳定的状态。
总体而言,网络重要节点检测算法主要从网络的拓扑结构出发,通过节点的中心性排序,来获得重要节点 \cite{Lue2016a}。
这里将重要节点检测算法大致分为以下4类:基于节点近邻的算法、基于网络路径的算法、基于特征向量的算法和其他算法 \cite{任晓龙2014}。

1)基于节点近邻的算法。这部分算法主要根据网络节点周围邻居节点的个数,节点所处的位置所构造,包括节点的度中心性算法、K-shell分解算法等。
K-shell分解算法是将网络外层的节点层层剥去,处在内层的节点核数更高,影响能力更大。该算法是Kitsak等人在研究网络信息传播问题时提出的算法,他们发现网络中最有效的传播者可以由K-shell分解方法所确定 \cite{Kitsak2010}。
Lu等人将H-index引入网络科学,提出了H算子,并证明在无权无向网络中,该H算子多次作用会使节点的度数收敛到节点核数;在收敛过程中,得到的一系列H指标同样可以用来衡量网络节点的重要性 \cite{Lue2016}。

2)基于网络路径的算法。这部分算法主要依据节点的路径规划,路径中节点的个数所构造,包括介数中心性算法、接近中心性算法、Katz中心性算法等。
节点的介数中心性刻画了在网络所有节点对的最短路径中,通过该节点的次数。节点的介数中心性越高,代表该节点在网络最短路径中更为重要。
接近中心性则刻画了节点到其他节点最短路径距离的平均值。
Katz中心性算法则考虑到了网络所有节点对的最短路径和其他路径,并做了一个加权处理:短路径权重较高,长路径权重较低。
然而,katz方法需大量的实验来计算最优的权重衰减因子,时间复杂度极高 \cite{Benzi2013}。

3)基于特征向量的算法。和基于网络节点的算法相似,这部分算法考虑到网络邻居节点数量的同时,还考虑了邻居节点质量。某个节点的邻居节点质量越高,影响能力越强,则该节点的影响能力也就越强。该部分算法主要包括:特征向量中心性算法,PageRank,LeaderRank等等。
PageRank方法则通过马尔科夫概率转移矩阵的收敛性,来对网络节点进行排序。
Lu等人在PageRank基础上,提出LeaderRank方法,增加了一个背景节点与其他所有节点相连,形成强联通网络 \cite{Li2014}。相比较而言,LeaderRank在在线社交网络的节点重要性排序中表现良好 \cite{Su2015}。

4)其他算法。有一些算法难以被归类到以上三种算法中,包括节点的收缩和删除算法,基于博弈论的重要节点检测算法等。
Restrepo等人认为复杂网络邻接矩阵的最大特征值是几类动力学过程的关键量,根据网络节点和连边对最大特征值的影响,对网络节点重要性进行了定量描述 \cite{Restrepo2006}。
Rao等人通过比较全连接网络的拓扑结构,提出了一种基于平均等效最短路径的网络无损评估方法,同时基于节点破坏后无损性下降百分比,来确定节点的重要性 \cite{Rao2009}。
Jia等人在节点收缩法的基础上,提出了节点间连边重要性衡量,将节点重要性定义为节点本身和连边重要性的加权和 \cite{Jia-sheng2011}。
Wang等人考虑到网络上的社区结构因素,以合作博弈理论为基础,通过Owen值得到网络上每个节点的边际贡献,从而来进行节点排序 \cite{王学光2013}。

\subsection{主要研究内容与意义}
本文着重于网络重要节点检测算法在牵制控制中的应用研究。
从大规模网络的控制问题出发,结合真实网络数据,分析了复杂网络的牵制控制问题,并提出了一个复杂度较低的网络重要节点检测算法。
具体的研究内容主要分为以下两个方面:

(1) 分析经典控制问题,提出耦合强度范围和收敛速度两个指标。
这部分主要针对于经典比率判据无法精确衡量网络牵制控制这一问题,提出了两个评价指标:耦合强度范围和收敛速度。
耦合强度范围衡量了网络受控时,耦合强度满足的条件。
收敛速度定义为网络受控达到稳定的速度,与系统的最大李雅普诺夫指数有关。
一般而言,耦合强度范围越大,代表网络更容易受到控制;网络收敛速度越小,代表系统中有一个更小的负李雅普诺夫指数,网络也就能更快地到达稳定状态。
本文在理论上分析了这三个指标的不同,在实验中验证了这三者无法同时到达最优状态
。

(2) 基于耦合强度范围扰动的重要节点检测算法。
传统重要节点检测算法在网络牵制控制节点的选取上,往往效果不够稳定,在不同的网络中效果差异巨大。这部分主要针对于传统算法效果不理想这一情况,基于提出的耦合强度范围指标,通过扰动理论,提出了一个重要节点检测算法。
该算法时间复杂度低,能够高效地计算出网络的重要节点。
本文在6个真实网络的实验中,通过与其他传统算法的对比,验证了基于耦合强度范围扰动算法的有效性。


耦合强度范围和收敛速度细化了复杂网络上牵制控制的评价指标,加深了对牵制控制的理解;
基于耦合强度范围扰动的重要节点检测算法在工程上部署控制节点时,有比较强的指导意义。

\subsection{论文的组织结构}
本文分为五章,文章的组织结构如下:

第一章:绪论。主要介绍了牵制控制的背景,网络重要节点检测算法的国内外研究状况,以及主要研究内容和意义。

第二章:网络科学基础理论。该部分主要介绍了复杂网络的基本概念,网络的同步模型和牵制控制模型以及经典重要节点检测算法。

第三章:网络的牵制控制问题。该部分主要介绍了衡量网络牵制控制能力的传统比率判据,耦合强度范围以及收敛速度三个指标,并给出其数学推导以及理论解释。
本章说明了这三个指标的差异;同时实验发现,三个指标无法同时到达最优状态。

第四章:重要节点检测算法在牵制控制上的应用研究。本章基于第三章提出的耦合强度范围这一指标,根据每个节点产生的耦合强度范围的扰动,提出了一种贪婪算法。
该算法复杂度低,能够有效地计算出网络的重要节点。在6个真实网络的实验中,通过与5种基准算法的对比,验证了基于耦合强度范围扰动算法的有效性。

第五章:总结与展望。本章概述了论文的主要工作以及论文的创新点,并总结了论文中所发现的一些问题,指明了未来可继续深入研究的领域和方向。

\clearpage
